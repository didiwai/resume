% -*- coding:utf-8 -*-
%% start of file `template_en.tex'.
%% Copyright 2006-1008 Xavier Danaux (xdanaux@gmail.com).
%
% This work may be distributed and/or modified under the
% conditions of the LaTeX Project Public License version 1.3c,
% available at http://www.latex-project.org/lppl/.
\documentclass[10pt,a4paper,sans]{moderncv}   % possible options include font size ('10pt', '11pt' and '12pt'), paper size ('a4paper', 'letterpaper', 'a5paper', 'legalpaper', 'executivepaper' and 'landscape') and font family ('sans' and 'roman')

\usepackage{fontspec,xunicode}
\setmainfont[BoldFont=Microsoft YaHei]{SimSun}
\usepackage[slantfont,boldfont]{xeCJK}
\usepackage{xcolor}                 % replace by the encoding you are using
\setCJKmainfont[BoldFont=Microsoft YaHei]{SimSun}
% moderncv 主题
\moderncvstyle{classic}                        %


\usepackage{booktabs}

% 调整页面出血
\setlength{\textheight}{30.5cm}
%\setlength{\parskip}{0\baselineskip}
\usepackage[scale=0.86]{geometry}
\setlength{\hintscolumnwidth}{2.8cm}           % 如果你希望改变日期栏的宽度

\linespread{1}


\firstname{姚}
\familyname{明}
\mobile{018637367007(非哈尔滨本地号码)}                    % optional, remove the line if not wanted
\email{yaoming147@163.com}                     % optional, remove the line if not wanted
\extrainfo{github.com/yaomingshr}
\address{出生年月:1991/6}{哈尔滨工业大学综合楼321室}
\photo[60pt]{yaoming.png}                         % '64pt' is the height the picture must be resized to and 'picture' is the name of the picture file; optional, remove the line if not wanted
%\nopagenumbers{}   
% to show numerical labels in the bibliography; only useful if you make citations in your resume
\makeatletter
\renewcommand*{\bibliographyitemlabel}{\@biblabel{\arabic{enumiv}}}
\makeatother
% uncomment to suppress automatic page numbering for CVs longer than one page
%----------------------------------------------------------------------------------
%            content
%----------------------------------------------------------------------------------
\begin{document}
\maketitle

\section{教育背景}
\cventry{2009.9-2013.7}{计算机科学与技术学士}{哈尔滨工业大学(HIT.)}{山东威
  海}{担任班长兼团支书}{}
\cventry{2013.7 至今}{计算机科学与技术硕士}{哈尔滨工业大学(HIT.)}{黑龙江哈
  尔滨}{}{}
\cvcomputer{英语水平}{CET6}{}{}{}{}
\section{竞赛、科研与项目经历}
\cventry{2013.11-2014.1}{Kaggle: Dogs vs. Cats}{}{}{}{}
\cvlistitem{项目描述:设计算法来分类图片中是狗还是猫}
\cvlistitem{项目地址:https://github.com/yaomingshr/DogvsCat}
\cvlistitem{数据描述:训练数据:580MB/25,000张 图片,测试数据:290MB/12,500 张图片}
\cvlistitem{项目实现概要:利用SIFT来提取特征,获取特征后,采用了Bag-of-Words + SVM。在BOW模型中,在对所有图片的特征点进行KMeans时,数据量太大,内存无法容
  下,所以将训练数据和测试数据分别分为3份,然后利用分别进行KMeans。BOW模型计算完
  毕之后,利用SVM进行分类。最后,利用随机森林对结果进行合并。}

\cventry{2014.2-2014.3}{Kaggle: Loan Default Prediction - Imperial College
  London}{}{}{}{}
\cvlistitem{项目描述:利用一笔贷款的相关信息预测其是否会违约;若违约,预测银行损
  失}
\cvlistitem{项目地址:https://github.com/lavizhao/loan}
\cvlistitem{训练数据: 510MB/10W+条 数据,测试数据:1.0GB/21W+条 数据; 每条数据
  有778个特征属性}
\cvlistitem{项目实现概要:主要包括数据预处理、分类和回归三部分。数据预处理部分,
  我们采用规则与统计检验方法(包括互信息和ROC曲线的AUC值)相结合的方法来筛选出对分类最起作用的特征。利用这些特
  征采用各种分类器(尝试了包括KNN、SVM以及随机森林等方法)来进行分类,判断该笔贷
  款到底是否会违约,分类结果较好时,再进一步进行回归,即预测违约贷款的损失值。}

\cventry{2013.3-2013.6}{基于轮廓跟踪的CT图像中骨骼边缘检测算法}{}{}{}{}
\cvlistitem{项目描述:在轮廓跟踪算法的基础上进行改进,扬长避短,获得更好的
  骨骼边缘}
\cvlistitem{项目地址:https://github.com/yaomingshr/contour\_trace}
\cvlistitem{项目实现概要:首先,实现了基础的轮廓跟踪算法。轮廓跟踪的优
  点在于保证轮廓完整闭合,但是抗噪性较差。之后尝试了利用跟踪虫等方法对轮廓跟踪的
  结果进行改进,效果一般。由于在CT图像中的噪声被普遍认定为高斯加性噪
  声,所以采用高斯函数的一阶导数的方法来获取法线方向同时抑制噪声是一种常见方法。
  我在此便是利用高斯一阶导来修正轮廓跟踪的搜索方向,获得了较好的结果。}

\section{获奖情况}
\cvcomputer{2012年}{全美大学生数学建模二等奖}{2013年}{山东省优秀毕业生}
\cvcomputer{2012年}{哈工大优秀团干部}{2010年}{哈工大优秀团员}
\cvcomputer{本科期间}{多次获得人民奖学金}{}{}

\section{专业技能}
%\cvcomputer{编程语言}{C/C++/python/shell}{操作系统}{Linux}
%\cvcomputer{编程语言}{python}{操作系统}{Linux}
%\cvcomputer{兴趣}{海量数据存储}{}{}
\cvlistitem{了解C/C++,熟悉Python等编程语言,使用过Linux操作系统}
\cvlistitem{掌握基本监督学习、无监督学习方法,了解基本算法与数据结构}
\cvlistitem{了解常见的数字图像处理方法}

%\cvlistitem{Item 1}
%\cvlistitem{Item 2}
%\cvlistitem[+]{Item 3}            % optional other symbol
\end{document}
%% end of file `template_en.tex'.
